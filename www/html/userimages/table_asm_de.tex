\documentclass[portrait, a0, final]{a0poster}
%%%%%%%%%%%%%
%% \tiny, \scriptsize, \footnotesize, \small, \normalsize, \large, \Large,
%% \LARGE, \huge, \Huge, \veryHuge, \VeryHuge, \VERYHuge 
%%%%%%%%%%%%
%% Fuer weitere Hinweise zu "a0poster" siehe Dokumentation "a0.tex".
%%
\usepackage{booktabs,units}
\usepackage{german}
\usepackage{amsmath, amssymb}  %AMS-Stuff
\usepackage[latin1]{inputenc}
\usepackage{exscale}
\usepackage{multicol}
\usepackage{helvet}
\usepackage{ifthen}
\usepackage{fancybox}
\usepackage{pifont}
\usepackage[dvips]{color}
%%
\definecolor{jwg}{rgb}{1,0.64705882353,0} % jwg -> das Orange der JWG-Uni
\definecolor{grau}{rgb}{0.9,0.9,0.9}
\definecolor{hellgrau}{rgb}{0.935,0.935,0.935}
\definecolor{gelb}{rgb}{1,1,0.7}
\definecolor{rot}{rgb}{1,0.875,0.935}
\definecolor{blau}{rgb}{0.75,0.91,1}
\definecolor{gruen}{rgb}{0.82,1,0.82}
\definecolor{webfontcolor}{rgb}{0.1922,0.2471,0.3412}
%% Ende der Farbdefinitionen.
\usepackage{epsfig}
\usepackage{psfrag}
\usepackage{array}
\usepackage{calc}
\usepackage{wrapfig}
\usepackage{subfigure}
%% ---------------------------------------------------------
%% - Verschieben des linken Rands, um den Text zu zentrieren (Vorschlag: -1.0cm)
\setlength{\oddsidemargin}{-1.0cm}
%%   Anpassen der Textbreite:
\addtolength{\textwidth}{1cm}  %%%{1cm}
%%   Anpassen der Zeilenbreite:
\addtolength{\linewidth}{1cm}
%% - Ein Trick, um Seitenumbrueche zu vermeiden
%\setlength{\textheight}{120cm}
%% Ende des Konfiguration der Seitenlayouts.
\setlength{\parindent}{0pt}
%%
\sloppy
%%%%%%%\frenchspacing
%% \frenchspacing unterdrueckt den Zusatzzwischenraum nach einem Satzzeichen
%% im nachfolgenden Text.
%%%%%%%%%\selectlanguage{german}
\selectlanguage{english}


\renewcommand{\baselinestretch}{1.2}


\begin{document}
%%
\sf
%% Der Befehl \sf waehlt eine serifenlose Schrift fuer das Poster.
%%%\large
\normalsize
%\large
\begin{minipage}{35cm}
\textcolor{webfontcolor}{
%%%%%%%%%%%%%%%%%%%%%%%%%%%%%%%%%%%%%%%%%%%%%%%%%%%%%%%%%%%%%
\begin{tabular}{ll}
\toprule Parameter & Wert\\ 
\midrule 
Gl\"attungsbreite im Ort $\sigma$ & \unit[600]{m}\\ 
Gl\"attungsbreite in der Zeit $\tau$  & \unit[40]{s} \\ 
Geschwindigkeit von St\"orungen im freien Verkehr $c_\text{free}$ & \unit[50]{km/h}\\
Geschwindigkeit von St\"orungen im gestauten Verkehr $c_\text{cong}$ & \unit[-15]{km/h}\\
\"Ubergangsschwelle vom freien in gestauten Verkehr $V_\text{thr}$ & \unit[58]{km/h}\\
\"Ubergangsbreite vom freien in gestauten Verkehr $\Delta V$ & \unit[5]{km/h}\\
\bottomrule
\end{tabular}
}
\end{minipage}
\end{document}
